\documentclass[conference]{IEEEtran}
\IEEEoverridecommandlockouts
% The preceding line is only needed to identify funding in the first footnote. If that is unneeded, please comment it out.
%Template version as of 6/27/2024

\usepackage{cite}
\usepackage{amsmath,amssymb,amsfonts}
\usepackage{algorithmic}
\usepackage{graphicx}
\usepackage{textcomp}
\usepackage{xcolor}
\def\BibTeX{{\rm B\kern-.05em{\sc i\kern-.025em b}\kern-.08em
    T\kern-.1667em\lower.7ex\hbox{E}\kern-.125emX}}

\begin{document}

\title{Multi-Target Trading Automation with Deep Q Learning}

\author{
    \IEEEauthorblockN{Yan-Fu Chen}
    \IEEEauthorblockA{
        \textit{Dept. of Computer Science} \\
        \textit{National Tsing-Hua University}\\
        Hsinchu, Taiwan \\
        aaaronyanfu@gmail.com
    }
    \and
    \IEEEauthorblockN{Sheng-You Chien}
    \IEEEauthorblockA{
        \textit{Dept. of Computer Science} \\
        \textit{National Tsing-Hua University}\\
        Nantou, Taiwan \\
        s99086tobby@gmail.com
    }
    \and
    \IEEEauthorblockN{Yi-Ning Chang}
    \IEEEauthorblockA{
        \textit{Dept. of Computer Science} \\
        \textit{National Tsing-Hua University}\\
        Hsinchu, Taiwan \\
        changyn@gapp.nthu.edu.tw
    }
    \and
    \IEEEauthorblockN{Jie-Hung Chen}
    \IEEEauthorblockA{
        \textit{Dept. of Computer Science} \\
        \textit{National Tsing-Hua University}\\
        Kaoshiung, Taiwan \\
        jiehongchen726@gmail.com
    }
    \and
    \IEEEauthorblockN{Yi-Hsueh Chu}
    \IEEEauthorblockA{
        \textit{Dept. of Computer Science} \\
        \textit{National Tsing-Hua University}\\
        Taoyuan, Taiwan \\
        ethan111062332@gapp.nthu.edu.tw
    }
    \and
    \IEEEauthorblockN{Bo-Yi Mao}
    \IEEEauthorblockA{
        \textit{Dept. of Computer Science} \\
        \textit{National Tsing-Hua University}\\
        Taipei, Taiwan \\
        dogeon188@gapp.nthu.edu.tw
    }
}

\maketitle

\begin{abstract}

% 毛

\end{abstract}

\begin{IEEEkeywords}
deep reinforcement learning, deep Q-network, stock trading, algorithmic trading
\end{IEEEkeywords}

\section{Introduction}

An automatic, algorithmic way of profit generation in the stock market has long been a persuit of many. With the rise of machine learning and deep learning, the possibility of using these technologies to predict stock prices and make trading decisions has become a reality.

% 毛

\section{Related Works}

% 朱 毛

\section{Methodology}

\subsection{Multi-Target Trading Architecture}

% 甫 桀

\subsection{Deep Neural Network}

% 甫 桀

\section{Experiments}

\subsection{Dataset}

% 朱 毛

\subsection{Evaluation Metrics}

% 張 簡

\subsection{Results}

% 張 簡

\section{Conclusion}

% 毛

\section{Data and Code Availability}

% 毛

\section{Author Contribution Statements}

% 毛

\begin{thebibliography}{0}
 
\bibitem{customenv}
    A. Gogikar, \emph{Multi-stock Algo-Trading environment for RL,} on Medium, Jun. 06, 2022. https://medium.com/@akhileshgogikar/custom-gym-environment-for-multi-stock-algo-trading-113b07dd445d (accessed Dec. 23, 2024).

\end{thebibliography}

\end{document}
