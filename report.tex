\documentclass[conference]{IEEEtran}
\IEEEoverridecommandlockouts
% The preceding line is only needed to identify funding in the first footnote. If that is unneeded, please comment it out.
%Template version as of 6/27/2024

\usepackage{cite}
\usepackage{amsmath,amssymb,amsfonts}
\usepackage{algorithmic}
\usepackage{graphicx}
\usepackage{textcomp}
\usepackage{xcolor}
\def\BibTeX{{\rm B\kern-.05em{\sc i\kern-.025em b}\kern-.08em
    T\kern-.1667em\lower.7ex\hbox{E}\kern-.125emX}}

\begin{document}

\title{Multi-Target Trading Automation with Deep Q Learning}

\author{\IEEEauthorblockN{Yan-Fu Chen}
\IEEEauthorblockA{\textit{Dept. of Computer Science} \\
\textit{National Tsing-Hua University}\\
Hsinchu, Taiwan \\
aaaronyanfu@gmail.com}
\and
\IEEEauthorblockN{Sheng-You Chien}
\IEEEauthorblockA{\textit{Dept. of Computer Science} \\
\textit{National Tsing-Hua University}\\
Nantou, Taiwan \\
s99086tobby@gmail.com}
\and
\IEEEauthorblockN{Jie-Hung Chen}
\IEEEauthorblockA{\textit{Dept. of Computer Science} \\
\textit{National Tsing-Hua University}\\
Kaoshiung, Taiwan \\
jiehongchen726@gmail.com}
\and
\IEEEauthorblockN{Yi-Hsueh Chu}
\IEEEauthorblockA{\textit{Dept. of Computer Science} \\
\textit{National Tsing-Hua University}\\
Taoyuan, Taiwan \\
ethan111062332@gapp.nthu.edu.tw}
\and
\IEEEauthorblockN{Yi-Ning Chang}
\IEEEauthorblockA{\textit{Dept. of Computer Science} \\
\textit{National Tsing-Hua University}\\
Hsinchu, Taiwan \\
changyn@gapp.nthu.edu.tw}
\and
\IEEEauthorblockN{Bo-Yi Mao}
\IEEEauthorblockA{\textit{Dept. of Computer Science} \\
\textit{National Tsing-Hua University}\\
Taipei, Taiwan \\
dogeon188@gapp.nthu.edu.tw}
}

\maketitle

\begin{abstract}
In this paper, we propose a novel approach to stock trading using a deep reinforcement learning model. We use a deep Q-network (DQN) to learn the optimal trading strategy. The model is trained on historical stock data and is able to make trading decisions based on the current stock price. We evaluate the performance of our model on a dataset of stock prices and compare it to a baseline model. Our results show that... We believe that our approach has the potential to revolutionize the way stock trading is done and shed light on the future of algorithmic trading.
\end{abstract}

\begin{IEEEkeywords}
deep reinforcement learning, deep Q-network, stock trading, algorithmic trading
\end{IEEEkeywords}

\section{Introduction}
An automatic, algorithmic way of profit generation in the stock market has long been a persuit of many. With the rise of machine learning and deep learning, the possibility of using these technologies to predict stock prices and make trading decisions has become a reality.

\section{Related Works}

\section{Methodology}

\subsection{Problem Formulation}

\subsection{Agent Model Architecture}

\subsection{Training Environment}

\section{Experiments}

\subsection{Dataset}

\subsection{Evaluation Metrics}

\subsection{Results}

\section{Discussion}

\section*{Acknowledgment}

The authors would like to thank...

\section*{References}

% \begin{thebibliography}{00}

% \end{thebibliography}

\end{document}
